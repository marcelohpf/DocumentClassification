\chapter[Introdução]{Introdução}

Técnicas de Aprendizado de Máquina (do inglês \textit{Machine Learning} ML) têm sido amplamente adotadas no mundo da computação para realizar tarefas em que há muitos dados disponíveis e existe algum tipo de relação entre eles \cite{brink_real-world_2015}.

Pode-se dizer que o ML é uma vertente da Inteligência Artificial (do inglês \textit{Artificial Intelligence} AI), o qual possui um corpo de conhecimento específico ou um conjunto de técnicas para que a máquina possa aprender com os dados. Enquanto que a AI é um campo de estudo muito mais abrangente que engloba os campos de visão computacional, processamento de linguagem natural, robótica e outros \cite{brink_real-world_2015}. 

Outro campo dentro da AI é o Processamento de Linguagem Natural (do inglês \textit{Natural Language Processing} NLP), o qual é extremamente importante para aplicações modernas, pois trata-se da interpretação dos constructos da linguagem humana para serem processadas por computadores \cite{goldberg_neural_2017}.
Com a digitalização dos documentos, viabilizou-se o uso de técnicas, métodos para facilitar análises e extração de informações de conteúdos \cite{oliveira_automatic_2017}.

Alguns dos conjuntos de técnicas para o NLP são o diálogo ou sistemas de fala, análises textuais e recuperação de informações através de perguntas e respostas \cite{eslick_langutils:_2005}. As técnicas, que são úteis para lidar com documentos digitais e facilitar a separação de conteúdo destes, são a classificação de documentos, busca e recuperação de informações \cite{oliveira_automatic_2017}.

Ainda que haja um grande esforço para pesquisa nas áreas específicas de NLP, é inerente a todas elas as dificuldades de fazer com que a máquina consiga lidar com a enorme variabilidade da linguagem natural, ambiguidade e usos complexos da língua como figuras de linguagens \cite{goldberg_neural_2017}. 

O Supremo Tribunal Federal (STF) é um órgão público da esfera de Poder Jurídico do Brasil. A ele compete realizar a guarda da Constituição (\citeyear{brasil_constituicao_1988}), no qual julga casos em que os Artigos da Constituição possam ser violados ou mal interpretados, conflitos entre a União e os Estados incluindo o Distrito Federal, ações movidas por estados estrangeiros, razões contra o Conselho Nacional de Justiça (CNJ), conflitos entre os Tribunais de Justiça \cite{brasil_constituicao_1988}.

Cabe a ele julgar em recurso ordinário crimes políticos, \textit{habeas corpus}, mandados de segurança e, em recurso extraordinário, infrações a Constituição, inconstitucionalidades em tratados ou em leis, invalidar atos do governo caso contrariem a Constituição \cite{brasil_constituicao_1988}.

O Governo Brasileiro implantou a política de transparência pública, no qual todas as informações públicas devem ser disponibilizadas para que os cidadão pudessem acessar qualquer informação sobre as três esferas do poder: Judiciário, Executivo e Legislativo \cite{noauthor_lei_2011}.

O sistema judiciário, assim como os outros dois poderes, teve um esforço para implantar todo o acesso a informação por meios eletrônicos, visto que a população em geral, quando quer buscar algo, utiliza os sites de busca como o Google \footnote{Site :\url{https://google.com}} e DuckDuckGo \footnote{Site: \url{https://duckduckgo.com}} \cite{ruschel_governo_2011}.

Logo então, o foco do sistema judiciário brasileiro foi no desenvolvimento dos \textit{web site} de cada tribunal e na digitalização dos processos. O CNJ, que  tem como uma de suas competências coordenar e auxiliar a digitalização de todos os 91 Tribunais de Justiça \cite{ruschel_governo_2011}, colocou como metas para o desenvolvimento do parque tecnológico: a informatização todas as unidades judiciárias, interligação e implantação do processo eletrônico em parcela  de  suas unidades judiciárias. %(CNJ metas 2009)
Com estas metas, esperava-se que todos os tribunais pudessem realizar a tramitação de processos por meio digital \cite{noauthor_termo_2009}. Também, que o processo fosse representado da mesma forma em diferentes sistemas, para que a população pudesse ter acesso a eles através de mecanismos de consulta online \cite{ruschel_governo_2011}.

Com isso, surgiu uma grande variabilidade de sistemas \textit{online} %(https://www.conjur.com.br/2017-out-03/excesso-sistemas-processo-eletronico-atrapalham-advogados)
mesmo com esforços do CNJ para realizar a unificação com o Modelo Nacional de Interoperabilidade \cite{noauthor_termo_2009}%(http://www.cnj.jus.br/busca-atos-adm?documento=2492)
. Muitos tribunais usam diferentes sistemas como PJe, Projudi e e-SAJ.

%% Verificar a necessidade deste parágrafo
O Processo Judicial eletrônico (PJe) foi um dos sistemas desenvolvidos, o qual passou a ser adotado por diversos tribunais como uso obrigatório. O seu principal objetivo é manter um sistema que possibilite a transição dos processos, bem como o registro de peças processuais, além do acompanhamento de todos os acontecimentos independentemente do tribunal em que ele tramite. A proposta de sua solução é que ele fosse único para todos os tribunais \cite{noauthor_wiki_2018}.

%% Encontrar alguma referência sobre isso
O fato desses sistemas seguirem a lógica de processos físicos, quanto a classificação do tipo de peça jurídica, a montagem de volumes, à forma de peticionamento, ao processo de tramitação e aos meta-dados, causa problemas relacionados a dificuldade de comunicação entre sistemas, armazenamento de arquivos, reclamações quanto a usabilidade do sistema e a recuperação de informações dos dados torna-se mais difícil.

Esses sistemas possibilitam a tramitação de processos digitalizados, os quais estão sujeitos a problemas de identificação de texto, furos, manchas, desalinhamento da página e suas incorretas ordenações de conteúdos. A Figura \ref{fig:bad_scan} exibe uma página de um desses processos, a qual apresenta problema de parte do escaneamento estar espelhado, fazendo com que o texto fique no sentido contrário.

\begin{figure}[ht]
	\centering
		\includegraphics[keepaspectratio=true,scale=0.3]{figuras/badScan}
	\caption[Página escaneada dentro de um volume]{Certidão de despacho digitalizada que possui parte de seu documento espelhado. Nomes pessoais e matrículas foram omitidos. Fonte: elaboração própria.}
	\label{fig:bad_scan}

\end{figure}

O Projeto Victor do STF tem o propósito de ser uma ferramenta que utiliza Inteligência Artificial para classificar novos processos em Repercussões Gerais (RG) \cite{supremo_tribunal_federal_inteligencia_2018}. Esse projeto é focado na função principal do STF, que é a garantia da constitucionalidade das Leis Federais em conformidade com a Constituição Federal (\citeyear{brasil_constituicao_1988}) \cite{brasil_constituicao_1988}. É um projeto de parcerias públicas entre STF e Universidade de Brasília, sendo o projeto de AI mais relevante para o contexto jurídico brasileiro \cite{supremo_tribunal_federal_inteligencia_2018}. 


\section{Problema e motivação}\label{sec:problema}

Como parte colaborativa para as pesquisas do Projeto Victor \cite{supremo_tribunal_federal_inteligencia_2018}, este trabalho ajuda nas análises de mérito de um processo do STF, nas análises de admissibilidade, ajuda na distrição de processos para os gabinetes do Tribunal, no direcionamento das partes dos processos aos acessores e possibilita facilitar o trabalho na maior corte brasileira.

O STF é um órgão público central, lida com os casos de toda a Federação, advindos dos tribunais e juizados da segunda instância \cite{brasil_constituicao_1988}. Ele tem que lidar com as diferentes padronizações de marcadores nos volumes de processos e código das peças. Além disso, os servidores avaliam apenas as peças essenciais a sua atividade, por conseguinte, precisam encontrar entre os diversos documentos do processo os que serão utilizados por eles.

Os próprios advogados podem fazer a classificação do tipo de peça ao submeter no sistema eletrônico do STF, desde que obedeçam o tamanho máximo de 10 MB por PDF\footnote{Formato de arquivo criado pela empresa Adobe Systems Incorporated para unificar o compartilhamento de conteúdo. Disponível em: \url{https://acrobat.adobe.com/br/pt/acrobat/about-adobe-pdf.html}. Acesso em: 17-06-2018} \cite{noauthor_resolucao_2010}. Por conta disso, há documentos que estão com apenas uma categoria carregando o conteúdo de várias peças.

Desta forma, foram identificados os seguintes problemas enfrentados pelo STF ao lidar com as peças de processos:

\begin{itemize}
	\item Processos advindos de diferentes fontes possuem códigos identificadores de peças e marcadores de volumes não padronizados, fato que dificulta aos servidores de encontrar rapidamente as de seu interesse.
    \item Alguns marcadores nos volumes são errados ou incompletos, ou seja, não trazem informação suficiente para caracterizar uma peça específica.
    \item Peças duplicadas produzidas pela ressubmissão de um processo, por conta da adição do recurso de admissibilidade. Com esta duplicação, os servidores têm dificuldade de encontrar informações de interesse rapidamente.
\end{itemize}

Diante da problemática apresentada, definiu-se a pergunta de pesquisa: Como identificar automatizadamente o corpo de texto de diferentes tipos de peças processuais jurídicas, que sejam analisadas pelo STF para repercussão geral, em um único documento?

\section{Hipótese}\label{sec:hipotese}

Foram desenvolvidas duas hipóteses \cite{gil_como_2002} para este trabalho baseadas na especificação do contexto e problemática do STF. 

\begin{itemize}
	\item As peças jurídicas avaliadas pelo STF para classificar um processo numa repercussão geral são computacionalmente separáveis.
    \item Não existem classes de peças diferentes com o mesmo conteúdo de texto.
\end{itemize}

\section{Objetivo}
Nesta seção, serão apresentados o objetivo geral e os objetivos específicos para a delimitação deste trabalho de conclusão de curso.

\subsection{Objetivo geral}

Classificar documentos jurídicos em diferentes categorias adotadas pelo STF utilizando técnicas de ML e NLP.

\subsection{Objetivos específicos}    

\begin{itemize}
  \item Levantar o estado da arte de classificação de documentos;
  \item Realizar análise exploratória dos dados;
  \item Construir um dicionário de palavras para o corpo textual dos processos.
  \item Elaborar arquitetura de pré-processamento dos dados;
  \item Realizar a transformação dos textos para representação computacional;
  \item Avaliar modelos para classificação de documentos;
  \item Modificar e evoluir modelos encontrados para ajustá-los ao problema do STF;
  \item Avaliar técnicas para validação de modelos.
  
\end{itemize}

\section{Organização do trabalho}
	Este trabalho estrá organizado em sete capítulos. O primeiro é a Introdução onde contextualiza-se a problemática e definem-se os objetivos, o segundo capítulo é para descrição da metodologia científica adotada. O terceiro é a fundamentação teórica para o desenvolvimento. O quarto é onde elabora-se uma breve análise dos dados. O quinto capítulo é onde apresenta-se os modelos propostos e os resultados de suas execuções. No sexto, faz-se uma discussão sobre os resultados obtidos e as referências adotadas. No sétimo e último, apresentam-se as conclusões do trabalho mediante todo o processo executado e os resultados coletados.