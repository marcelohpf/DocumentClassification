\chapter[Introdução]{Introdução}

Técnicas de machine learning tem sido amplamente adotadas no mundo da computação para realizar tarefas em que há muitos dados disponíveis e existem algum tipo de relação entre eles de forma que seja possível escrever um algorítmo para processá-lo.

Assim como NLP tem sido bastante utilizado para tarefas em que viabilizam ainda mais a integração entre a linguagem do homem para a máquina.

A tecnologia no sistema judiciário brasileiro.

\section{Problema e motivação}

DESCOBRIR AMANHÃ COM O PESSOAL DO STF

\section{Objetivos}
Nesta seção, serão apresentados o objetivo geral e os objetivos específicos para a delimitação deste trabalho de conclusão de curso.

\subsection{Objetivo geral}

Classificar peças jurídicas utilizando técnicas de inteligência artificial

\subsection{Objetivos específicos}    

\begin{itemize}
  \item Identificar técnicas para classificação de documentos
  \item Realizar análise exploratória dos dados
  \item Elaborar arquitetura de preprocessamento dos dados
  \item Escolher modelo de classificação para documentos
  \item Utilizar técnicas de validação para o modelo
\end{itemize}

\section{Organização do trabalho}
	Este trabalho estará organizado em X capítulos. O primeiro capítulo é a fundamentação teórica para o desenvolvimento deste trabalho. O segundo capítulo é para a metodologia científica do trabalho e descrição dos dados. O terceiro capítulo é onde elabora-se a arquitetura e solução. O quarto capítulo é onde discute-se os resultados da implementação da arquitetura proposta e no ultimo capítulo, apresenta-se a conclusão.