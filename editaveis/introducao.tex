\chapter[Introdução]{Introdução}

Técnicas de Aprendizado de Máquina (do inglês \textit{Machine Learning} ML) tem sido amplamente adotadas no mundo da computação para realizar tarefas em que há muitos dados disponíveis e existe algum tipo de relação entre eles. De forma que seja possível escrever um algoritmo para processá-lo \cite{BRINK2015}.

Pode-se dizer que o aprendizado de máquina é uma vertente da Inteligência Artificial (do inglês \textit{Artificial artificial intelligence} AI), a qual possui um corpo de conhecimento específico ou um conjunto de técnicas para que a máquina possa aprender com os dados. Enquanto que a AI é um campo de estudo muito mais abrangente que engloba os campos de visão computacional, processamento de linguagem natural, robótica e outros \cite{BRINK2015}. 

Outro campo dentro da AI é o Processamento de Linguagem Natural (do inglês \textit{Natural Language Processing} NLP), o qual é extremamente importante para aplicações modernas, pois trata-se da interpretação dos constructos da linguagem humana para serem processadas por computadores \cite{GOLDBERG2017}.
Com a digitalização dos documentos, viabilizou-se o uso de técnicas, métodos para facilitar análises e extração de informações de conteúdos \cite{OLIVEIRA2017}.

Alguns dos conjuntos de técnicas para o NLP são o diálogo ou sistemas de fala, análises textuais e recuperação de informações através de perguntas e respostas \cite{ESLICKandLIU2005}. As técnicas, que são úteis para lidar com documentos digitais e facilitar a separação de conteúdo destes, são a classificação de documentos, busca e recuperação de informações \cite{OLIVEIRA2017}.

Ainda que haja um grande esforço nas áreas específicas para as categorias de técnicas de NLP, é inerente a todas elas as dificuldades encontradas em fazer com que a máquina consiga lidar com a enorme variabilidade da linguagem natural, ambiguidade e usos complexos da língua como figuras de linguagens \cite{GOLDBERG2017}. 

O Supremo Tribunal Federal (STF) é um órgão público da esfera de Poder Jurídico do Brasil. A ele compete realizar a guarda da Constituição (\citeyear{BRASIL1988}), no qual julga casos em que os Artigos da Constituição possam ser violados ou mal interpretados, conflitos entre a União e os Estados incluindo o Distrito Federal, ações movidas por estados estrangeiros, razões contra o Conselho Nacional de Justiça (CNJ), conflitos entre os tribunais \cite{BRASIL1988}.

Além disso, cabe a ele julgar em recurso ordinário crimes políticos, \textit{habeas corpus}, mandados de segurança e, em recurso extraordinário, infrações a Constituição, inconstitucionalidades em tratados ou em leis, invalidar atos do governo caso contrariem a Constituição \cite{BRASIL1988}.

O Governo brasileiro implantou a política de transparência pública, no qual todas as informações públicas devem ser disponibilizadas para que os cidadão pudessem acessar qualquer informação sobre as três esferas do poder: Judiciário, Executivo e Legislativo \cite{CONGRESSO2011}.

O sistema judiciário, assim como os outros dois poderes, teve um esforço para implantar todo o acesso a informação por meios eletrônicos, visto que a população em geral, quando quer buscar algo, utiliza os sites de busca como o Google \footnote{Site :\url{https://google.com}} e DuckDuckGo \footnote{Site: \url{https://duckduckgo.com}} \cite{RUSCHEL2011}.

Logo então, o foco do sistema judiciário brasileiro foi no desenvolvimento dos \textit{web site} de cada tribunal e na digitalização dos processos. O CNJ  tem como uma de suas competências coordenar e auxiliar a digitalização de todos os 91 tribunais de justiça \cite{RUSCHEL2011}. Ele colocou como metas para o desenvolvimento do parque tecnológico, a de informatizar todas as unidades judiciárias, interligá-las e implantar  o  processo  eletrônico  em  parcela  de  suas unidades judiciárias. %(CNJ metas 2009)
Com estas metas, esperava-se que todos os tribunais pudessem realizar a tramitação de processos por meio digital \cite{CNJ2009}. Também, que o processo fosse representado da mesma forma em diferentes sistemas, para que a população pudesse ter acesso a eles através de mecanismos de consulta online \cite{RUSCHEL2011}.

Com isso, surgiu uma grande variabilidade de sistemas%(https://www.conjur.com.br/2017-out-03/excesso-sistemas-processo-eletronico-atrapalham-advogados)
. Mesmo com esforços do Conselho de Justiça para realizar a unificação com o Modelo Nacional de Interoperabilidade \cite{CNJ2009} e a adoção do sistema PJe%(http://www.cnj.jus.br/busca-atos-adm?documento=2492)
. Muitos tribunais usam diferentes sistemas como PJe, Projudi e e-SAJ.

%% Verificar a necessidade deste parágrafo
O Processo Judicial eletrônico (PJe) foi um dos sistemas desenvolvidos, o qual passou a ser adotado por diversos tribunais como uso obrigatório. O seu principal objetivo é manter um sistema que possibilite a transição dos processos, bem como o registro de peças processuais, além do acompanhamento de todos os acontecimentos independentemente do tribunal em que ele tramite. A proposta de sua solução é que ele fosse único para todos os tribunais \cite{PJe2018}.

%% Encontrar alguma referência sobre isso
Grande parte do problema envolto com a grande variabilidade destes sistemas e que dificultam sua implementação, é o fato destes sistemas ainda seguirem a lógica de processos físicos. Tal qual a classificação do tipo de peça jurídica, a montagem de volumes, a forma de peticionamento, a tramitação, os meta-dados, até o fato de que ainda existem muitos processos que são digitalizados. A Figura \ref{fig:bad_scan} exibe um desses processos, os quais são cópias digitais de processos físicos e que podem estar sujeitos a problemas de identificação do conteúdo, furos, manchas, desalinhamento da página e suas incorretas ordenações.

\begin{figure}[!ht]
	\centering
	\label{fig:bad_scan}
		\includegraphics[keepaspectratio=true,scale=0.3]{figuras/badScan}
	\caption{Certidão de despacho digitalizada em volume}
\end{figure}

\section{Problema e motivação}\label{sec:problema}

Como o STF é um órgão público central, lida com os casos de toda a Federação, advindos dos tribunais e juizados da segunda instância \cite{BRASIL1988}. Ele tem que lidar com as diferentes padronizações de marcadores nos volumes de processos e código das peças. Além disso, os servidores avaliam apenas as peças essenciais a sua atividade, por conseguinte, precisam encontrar entre os diversos documentos do processo os que serão utilizados por eles.

Os próprios advogados podem fazer a classificação do tipo de peça ao submeter no sistema eletrônico do STF, desde que obedeçam o tamanho máximo de 10 MB por PDF\footnote{Formato de arquivo criado pela empresa Adobe Systems Incorporated para unificar o compartilhamento de conteúdo. Disponível em: \url{https://acrobat.adobe.com/br/pt/acrobat/about-adobe-pdf.html}. Acesso em: 17-06-2018} \cite{STF2010}. Por conta disso, há documentos que estão com apenas uma categoria carregando o conteúdo de várias peças.
% * <marcelohpf@gmail.com> 2018-06-20T04:19:16.524Z:
%
% ^.

Desta forma, foram identificados os seguintes problemas enfrentados pelo STF ao lidar com as peças de processos:

\begin{itemize}
	\item Processos advindos de diferentes fontes possuem códigos identificadores de peças e marcadores de volumes não padronizados, fato o qual dificulta os servidores de encontrar rapidamente as de seu interesse.
    \item Alguns marcadores nos volumes são errados ou incompletos, ou seja, não trazem informação suficiente para caracterizar uma peça específica.
    \item Peças duplicadas produzidas pela ressubmissão de um processo, por conta da adição do recurso de admissibilidade. Com esta duplicação, os servidores têm dificuldade de encontrar informações de interesse rapidamente.
\end{itemize}

Diante da problemática apresentada, definiu-se a pergunta de pesquisa: Como identificar automatizadamente o corpo de texto de diferentes tipos de peças processuais jurídicas, que sejam analisadas pelo STF para repercussão geral, em um único documento?

\section{Hipóteses}

Foram desenvolvidas duas hipóteses \cite{GIL2002} para este trabalho baseadas na especificação do contexto e problemática do STF. 

\begin{itemize}
	\item As peças jurídicas avaliadas pelo STF para classificar um processo numa repercussão geral são computacionalmente separáveis.
    \item Não existem classes de peças diferentes com o mesmo conteúdo de texto.
\end{itemize}

\section{Objetivos}
Nesta seção, serão apresentados o objetivo geral e os objetivos específicos para a delimitação deste trabalho de conclusão de curso.

\subsection{Objetivo geral}

Classificar e indexar documentos jurídicos em categorias de peças processuais utilizando técnicas de ML e NLP.

\subsection{Objetivos específicos}    

\begin{itemize}
  \item Identificar técnicas para classificação de documentos;
  \item Realizar análise exploratória dos dados;
  \item Elaborar arquitetura de preprocessamento dos dados;
  \item Realizar a transformação dos textos para representação computacional;
  \item Classificar para documentos;
  \item Utilizar técnicas de validação para o modelo.
\end{itemize}

\section{Organização do trabalho}
	Este trabalho estará organizado em cinco capítulos. O primeiro é a Introdução do trabalho, o segundo capítulo é para a metodologia científica do trabalho. O terceiro é a fundamentação teórica para o desenvolvimento deste trabalho. O quarto é onde elabora-se uma breve análise dos dados e define propostas de soluções para o problema. O quinto capítulo é onde discute-se os resultados da implementação da solução proposta e que será desenvolvido apenas na continuação desta pesquisa.